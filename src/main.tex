% ---------------------------------------------------------------------
% -------------- PREAMBLE ---------------------------------------------
% ---------------------------------------------------------------------
\documentclass[12pt,a4paper,english,oneside]{article}
%\documentclass[12pt,a4paper,finnish,twoside]{article}
%\documentclass[12pt,a4paper,finnish,oneside,draft]{article} % luonnos, nopeampi

% Valitse 'input encoding':
%\usepackage[latin1]{inputenc} % merkistökoodaus, jos ISO-LATIN-1:tä.
\usepackage[utf8]{inputenc}   % merkistökoodaus, jos käytetään UTF8:a
% Valitse 'output/font encoding':
%\usepackage[T1]{fontenc}      % korjaa ääkkösten tavutusta, bittikarttana
\usepackage{ae,aecompl}       % ed. lis. vektorigrafiikkana bittikartan sijasta
% Kieli- ja tavutuspaketit:
\usepackage[english]{babel}
% Kurssin omat asetukset aaltosci_t.sty:
% \usepackage{aaltosci_t}
% Jos kirjoitat muulla kuin suomen kielellä valitse:
%\usepackage[finnish]{aaltosci_t}           
%\usepackage[swedish]{aaltosci_t}           
\usepackage[english]{aaltosci_t}           
% Muita paketteja:
\usepackage{alltt}
\usepackage{amsmath}   % matematiikkaa
\usepackage{calc}      % käytetään laskurien (counter) yhteydessä (tiedot.tex)
\usepackage{eurosym}   % eurosymboli: \euro{}
\usepackage{url}       % \url{...}
\usepackage{listings}  % koodilistausten lisääminen
\usepackage{algorithm} % algoritmien lisääminen kelluvina
\usepackage{algorithmic} % algoritmilistaus
\usepackage{hyphenat}  % tavutuksen viilaamiseen liittyvä (hyphenpenalty,...)
\usepackage{supertabular,array}  % useampisivuinen taulukko

% Koko dokumentin kattavia asetuksia:

% Tavutettavia sanoja:
%\hyphenation{vää-rin me-ne-vi-en eri-kois-ten sa-no-jen tavu-raja-ehdo-tuk-set}
% Huomaa, että ylläoleva etsii tarkalleen kyseisiä merkkijonoja, eikä
% ymmärrä taivutuksia. Paikallisesti tekstin seassa voi myös ta\-vut\-taa.

% Rangaistaan tavutusta (ei toimi?! Onko hyphenat-paketti asennettu?)
\hyphenpenalty=10000   % rangaistaan tavutuksesta, 10000=ääretön
\tolerance=1000        % siedetään välejä riveillä
% titlesec-paketti auttaa, jos tämän mukana menee sekaisin

% Tekstiviitteiden ulkoasu.
% Pakettiin natbib.sty/aaltosci.bst liittyen katso esim. 
% http://merkel.zoneo.net/Latex/natbib.php
% jossa selitykset citep, citet, bibpunct, jne.
% Valitse alla olevista tai muokkaa:
\bibpunct{(}{)}{;}{a}{,}{,}    % a = tekijä-vuosi (author-year)
%\bibpunct{[}{]}{;}{n}{,}{,}    % n = numero [1],[2] (numerical style)

% Rivivälin muuttaminen:
\linespread{1.24}\selectfont               % riviväli 1.5
%\linespread{1.24}\selectfont               % riviväli 1, kun kommentoit pois

% ---------------------------------------------------------------------
% -------------- DOCUMENT ---------------------------------------------
% ---------------------------------------------------------------------

\begin{document}

% -------------- Tähän dokumenttiin liittyviä valintoja  --------------

%\raggedright         % Tasattu vain vasemmalta, ei tavutusta
% ----------------- joitakin makroja ----------------------------------
%
% \newcommand{\sinunKomentosi}[argumenttienMäärä]{komennot%
% voiJakaaRiveille%
% jaArgumenttienViittaus#1,#2,#argumenttienMäärä}

% Joskus voi olla tarpeen kommentoida jotakin. Ei suositella. 
% Äläkä unohda lopulliseen! 
\newcommand{\Kommentti}[1]{\fbox{\textbf{COMMENT:} #1}}
% Käyttö: Kilometri on 1024 metriä. \Kommentti{varmista tämä vielä}.
% Eli newcommand:n komentosanan jälkeen hakasaluissa argumenttien lkm,
%  ja argumentteihin viitataa #1, #2, ...

%  Comment out this \DRAFT macro if this version no longer is one!  XXX
%\newcommand{\DRAFT}{\begin{center} {\it DRAFT! \hfill --- \hfill DRAFT!
%\hfill --- \hfill DRAFT! \hfill --- \hfill DRAFT!}\end{center}}

%  Use this \DRAFT macro in the final version - or comment out the 
%  draft-command
% \newcommand{\DRAFT}{~}

% %%%%%%%% MATEMATIIKKA %%%%%%%%%%%%%%%%%

% Määrätty integraali
\newcommand{\myInt}[4]{%
\int_{#1}^{#2} #3 \, \textrm{d}{#4}}

% http://kapsi.fi/jks/satfaq/
%\newcommand{\vii}{\mathop{\Big/}}
%\newcommand{\viiva}[2]{\vii\limits_{\!\!\!\!{#1}}^{\>\,{#2}}}
%%\[ \intop_0^{10} \frac{x}{x^2+1} \,\mathrm{d}x
%%= \viiva{0}{10} \frac{1}{2}\ln(x^2+1) \]

% matht.sty, Simo K. Kivelä, 01.01.2002, 07.04.2004, 19.11.2004, 21.02.2005
% Kokoelma matemaattisten lausekkeiden kirjoittamista helpottavia
% määrittelyjä.

% 07.04.2004 Muutama lisäys ja muutos tehty: \ii, \ee, \dd, \der,
% \norm, \abs, \tr.
%
% 19.11.2004 Korjattu määrittelyjä: \re, \im, \norm;
% lisätty \trp (transponointi), \hrm (hermitointi), \itgr (rakenteellinen
% integraali), ympäristö Cmatrix (hakasulkumatriisi);
% vanha transponointi \tr on mukana edelleen, mutta ei suositella.

% Pakotettu rivinvaihto, joka voidaan tarvittaessa määritellä
% uudelleen: 

%\newcommand{\nl}{\newline}

% Logiikan symboleja (<=> ja =>) hieman muunnettuina:

%\newcommand{\ifftmp}{\;\Leftrightarrow\;}
%\newcommand{\impltmp}{\DOTSB\;\Rightarrow\;}

% 'siten, että' -lyhenne ja hattupääyhtäläisyysmerkki vastaavuuden
% osoittamiseen: 

%\newcommand{\se}{\quad \text{siten, että} \quad}
%\newcommand{\vs}{\ {\widehat =}\ }

% Lukujoukkosymbolit:

%\newcommand{\N}{\ensuremath{\mathbb N}}
%\newcommand{\Z}{\ensuremath{\mathbb Z}}
%\newcommand{\Q}{\ensuremath{\mathbb Q}}
%\newcommand{\R}{\ensuremath{\mathbb R}}
%\newcommand{\C}{\ensuremath{\mathbb C}}

% Reaali- ja imaginaariosa, imaginaariyksikkö:

%\newcommand{\re}{\operatorname{Re}}
%\newcommand{\im}{\operatorname{Im}}
%\newcommand{\ii}{\mathrm{i}}

% Differentiaalin d, Neperin luku:

%\newcommand{\dd}{\mathrm{d}}
%\newcommand{\ee}{\mathrm{e}}

% Vektorimerkintä, joka voidaan tarvittaessa määritellä uudelleen
% (tämä tekee vektorit lihavoituina):

%\newcommand{\V}[1]{{\mathbf #1}}

% Kulmasymboli:

%\renewcommand{\angle}{\sphericalangle}

% Vektorimerkintä, jossa päälle pannaan iso nuoli;
% esimerkiksi \overrightarrow{AB} (tämmöisiä olemassaolevien
% symbolien uudelleenmäärittelyjä ei kyllä pitäisi tehdä):

%\renewcommand{\vec}[1]{\overrightarrow{#1}}

% Vektoreiden vastakkaissuuntaisuus:

%\newcommand{\updownarrows}{\uparrow\negthinspace\downarrow}

% Itseisarvot ja normi:

%\newcommand{\abs}[1]{{\left\vert#1\right\vert}}
%\newcommand{\norm}[1]{{\left\Vert #1 \right\Vert}}

% Transponointi ja hermitointi:

%\newcommand{\trp}[1]{{#1}\sp{\operatorname{T}}}
%\newcommand{\hrm}[1]{{#1}\sp{\operatorname{H}}}

% Vanha transponointi; jäljellä yhteensopivuussyistä, ei syytä käyttää.
%\newcommand{\tr}{{}^{\text T}}

% Arcus- ja area-funktiot, jossa päähaara osoitetaan nimen päälle
% vedetyllä vaakasuoralla viivalla (alkaa olla vanhentunutta,
% voitaisiin luopua):

%\newcommand{\arccot}{\operatorname{arccot}}
%\newcommand{\asin}{\operatorname{\overline{arc}sin}}
%\newcommand{\acos}{\operatorname{\overline{arc}cos}}
%\newcommand{\atan}{\operatorname{\overline{arc}tan}}
%\newcommand{\acot}{\operatorname{\overline{arc}cot}}

%\newcommand{\arsinh}{\operatorname{arsinh}}
%\newcommand{\arcosh}{\operatorname{arcosh}}
%\newcommand{\artanh}{\operatorname{artanh}}
%\newcommand{\arcoth}{\operatorname{arcoth}}
%\newcommand{\acosh}{\operatorname{\overline{ar}cosh}}

% Signum, syt, pyj:

%\newcommand{\sg}{\operatorname{sgn}}
%\renewcommand{\gcd}{\operatorname{syt}}
%\newcommand{\lcm}{\operatorname{pyj}}

% Lyhennemerkintöjä: derivaatta, osittaisderivaatta, gradientti,
% derivaattaoperaattori, vektorin komponentti, integraalin ylä- ja
% alasumma, Suomessa (ja Saksassa?) käytetty integraalin sijoitus-
% merkintä, integraali (rakenteellinen määrittely):

%\newcommand{\der}[2]{\frac{\dd #1}{\dd #2}}
%\newcommand{\osder}[2]{\frac{\partial #1}{\partial #2}}
%\newcommand{\grad}{\operatorname{grad}}
%\newcommand{\Df}{\operatorname{D}} 
%\newcommand{\comp}{\operatorname{comp}}
%\newcommand{\ys}[1]{\overline S_{#1}}
%\newcommand{\as}[1]{\underline S_{#1}}
%\newcommand{\sijoitus}[2]{\biggl/_{\null\hskip-6pt #1}^{\null\hskip2pt #2}} 
%\newcommand{\itgr}[4]{\int_{#1}^{#2}#3\,\dd #4}

% Matriiseja, joille voidaan antaa alkioiden sijoittamista sarakkeen
% vasempaan tai oikeaan reunaan tai keskelle osoittava lisäparametri
% (l, r tai c); ympärillä kaarisulut, hakasulut, pystyviivat (determinantti)
% tai ei mitään;
% esimerkiksi \begin{cmatrix}{ll}1 & -1 \\ -1 & 1 \end{cmatrix}:

%\newenvironment{cmatrix}[1]{\left(\begin{array}{#1}}{\end{array}\right)}
%\newenvironment{Cmatrix}[1]{\left[\begin{array}{#1}}{\end{array}\right]}
%\newenvironment{dmatrix}[1]{\left|\begin{array}{#1}}{\end{array}\right|}
%\newenvironment{ematrix}[1]{\begin{array}{#1}}{\end{array}}

% Kaunokirjoitussymboli:

%\newcommand{\Cal}{\mathcal}

% Isokokoinen summa:

%\newcommand{\dsum}[2]{{\displaystyle \sum_{#1}^{#2}}}

% Tuplaintegraali umpinaisen pinnan yli; korvataan jos parempi löytyy:
%\newcommand\oiint{\begingroup
% \displaystyle \unitlength 1pt
% \int\mkern-7.2mu
% \begin{picture}(0,3)
%   \put(0,3){\oval(10,8)}
% \end{picture}
% \mkern-7mu\int\endgroup}
       % Haetaan joitakin makroja

% Kieli:
% Kielesi, jolla kandidaatintyön kirjoitat: finnish, swedish, english.
% Tästä tulee mm. tietyt otsikkonimet ja kuva- ja taulukkoteksteihin 
% (Kuva, Figur, Figure), (Taulukko, Tabell, Table) sekä oikea tavutus.
\selectlanguage{english}

% Sivunumeroinnin kanssa pieniä ristiriitaisuuksia.
% Toimitaan pääosin lähteen "Kirjoitusopas" luvun 5.2.2 mukaisesti.
% Sivut numeroidaan juoksevasti arabialaisin siten että 
% ensimmäiseltä nimiölehdeltä puuttuu numerointi.
\pagestyle{plain}
\pagenumbering{arabic}
% Muita tapoja: kandiohjeet: ei numerointia lainkaan ennen tekstiosaa
%\pagestyle{empty}
% Muita tapoja: kandiohjeet: roomalainen numerointi alussa ennen tekstiosaa
%\pagestyle{plain}
%\pagenumbering{roman}        % i,ii,iii, samalla alustaa laskurin ykköseksi

% ---------------------------------------------------------------------
% -------------- Luettelosivut alkavat --------------------------------
% ---------------------------------------------------------------------

% -------------- Nimiölehti ja sen tiedot -----------------------------
%
% Nimiölehti ja tiivistelmä kirjoitetaan seminaarin mukaan joko
% suomeksi tai ruotsiksi (ellei erityisesti kielenä ole englanti). 
% Tiivistelmän voi suomen/ruotsin lisäksi kirjoittaa halutessaan
% myös englanniksi. Eli tiivistelmiä tulee yksi tai kaksi kpl.
%
% "\MUUTTUJA"-kohdat luetaan aaltosci_t.sty:ä varten.

\author{Peik Etzell}

% Otsikko nimiölehdelle. Yleensä sama kuin seuraavana oleva \TITLE, 
% mutta jos nimiölehdellä tarvetta "kaksiosaiselle" kaksiriviselle
\title{Coordinated Robotic Motion Planning: 
\\[5mm]An overview}
% 2-osainen otsikko:
%\title{\LaTeX{}-pohja kandidaatintyölle \\[5mm] Pitkiä rivejä kokeilun vuoksi.}

% Otsikko tiivistelmään. Jos lisäksi engl. tiivistelmä, niin viimeisin:
\TITLE{\LaTeX{}-pohja kandidaatintyötä varten ohjeiden kera ja varuilta %
kokeillaan vähän ylipitkää otsikkoa}
%\TITLE{\LaTeX{} för kandidatseminariet med jättelång rubrik som fortsätter och
% fortsätter ännu}
\ENTITLE{\LaTeX{} template for Bachelor thesis with a pretty long title %
line which continues ynd continues}
% 2-osainen otsikko korvataan täällä esim. pisteellä:
%\TITLE{\LaTeX{}-pohja kandidaatintyölle. Pitkiä rivejä kokeilun vuoksi.}

% Ohjaajan laitos suomi/ruotsi ja tarvittaessa eng (tiivistelmän kieli/kielet)
\DEPT{Poimi tähän ohjaajasi laitos, DEPT, main.tex}
% suomi:
%\DEPT{Tietotekniikan laitos}               % T
%\DEPT{Tietojenkäsittelytieteen laitos}     % TKT
%\DEPT{Mediatekniikan laitos}               % ME
% ruotsi:
%\DEPT{Institutionen för datateknik}        % T
%\DEPT{Institutionen för datavetenskap}     % TKT
%\DEPT{Institutionen för mediateknik}       % ME
% englanti:
%\ENDEPT{Department of Computer Science Engineering}     % T
%\ENDEPT{Department of Information and Computer Science} % TKT
%\ENDEPT{Department of Media Technology}                 % ME

% Vuosi ja päivämäärä, jolloin työ on jätetty tarkistettavaksi.
\YEAR{2023}
\DATE{xx. xxxxxkuuta 20xx}
%\DATE{31. helmikuuta 2011}
%\DATE{Den 31 februari 2011}
\ENDATE{MonthName 31, 20xx}

% Kurssin vastuuopettaja ja työsi ohjaaja(t)
\SUPERVISOR{Apulaisprofessori Juho Kannala}
\INSTRUCTOR{Ohjaajantitteli Sinun Ohjaajasi}
%\INSTRUCTOR{Ohjaajantitteli Sinun Ohjaajasi, ToinenTitt Matti Meikäläinen}
% DI       // på svenska DI diplomingenjör
% TkL      // TkL teknologie licentiat
% TkT      // TkD teknologie doctor
% Dosentti Dos. // Doc. Docent
% Professori Prof. // Prof. Professor
% 
% Jos tiivistelmä englanniksi, niin:
\ENSUPERVISOR{Assistant Professor Juho Kannala}
\ENINSTRUCTOR{Francesco d'Amore}
% M.Sc. (Tech)  // M.Sc. (Eng)
% Lic.Sc. (Tech)
% D.Sc. (Tech)   // FT filosofian tohtori, PhD Doctor of Philosophy
% Docent
% Professor

% Kirjoita tänne HOPS:ssa vahvistettu pääaineesi.
% Pääainekoodit TIK-opinto-oppaasta.

\PAAAINE{}
%Tietotekniikka
%Datateknik
%Computer Science and Engineering
\CODE{SCI3027}

%\PAAAINE{Ohjelmistotuotanto ja -liiketoiminta}
%\CODE{T3003}
%
%\PAAAINE{Tietoliikenneohjelmistot}
%\CODE{T3005}
%
%\PAAAINE{WWW-teknologiat} % vuodesta 2010
%\CODE{IL3012}
%
%\PAAAINE{Mediatekniikka} % vuoteen 2010, kts. seur.
%\CODE{T3004}
%
%\PAAAINE{Mediatekniikka} % vuodesta 2010, kts. edell.
%\CODE{IL3011}
%
%\PAAAINE{Tietojenkäsittelytiede} % vuodesta 2010
%\CODE{IL3010}
%
%\PAAAINE{Informaatiotekniikka} % vuoteen 2010
%\CODE{T3006}
%
%\PAAAINE{Tietojenkäsittelyteoria} % vuoteen 2010
%\CODE{T3002}
%
%\PAAAINE{Ohjelmistotekniikka}
%\CODE{T3001}

% Avainsanat tiivistelmään. Tarvittaessa myös englanniksi:

% \KEYWORDS{avain, sanoja, niitäkin, tähän, vielä, useampi, vaikkei, %
% niitä, niin, montaa, oikeasti, tarvitse}
\ENKEYWORDS{key, words, the same as in FIN/SWE}

% Tiivistelmään tulee opinnäytteen sivumäärä.
% Kirjoita lopulliset sivumäärät käsin tai kokeile koodia. 
%
% Ohje 29.8.2011 kirjaston henkilökunnalta:
%   - yhteissivumäärä nimiölehdeltä ihan loppuun
%   - "kaikkien yksinkertaisin ja yksiselitteisin tapa"
%
% VANHA // Ohje 14.11.2006, luku 4.2.5:
% VANHA // - sivumäärä = tekstiosan (alkaen johdantoluvusta) ja 
% VANHA //  lähdeluettelon sivumäärä, esim. "20"
% VANHA // - jos liitteet, niin edellisen lisäksi liitteiden sivumäärä,
% VANHA //  tyyli "20 + 5", jossa 5 sivua liitteitä 
% VANHA // - HUOM! Tässä oletuksena sivunumerointi alkaa nimiölehdestä 
% VANHA //  sivunumerolla 1. %   Toisin sanoen, viimeisen lähdeluettelosivun 
% VANHA //  sivunumero EI ole sivujen määrä vaan se pitää laskea tähän käsin

\PAGES{Kirjoita tähän oikea määrä, tässä esimerkissä 23}
%\PAGES{23}  % kaikki sivut laskettuna nimiölehdestä lähdeluettelon tai 
             % mahdollisten liitteiden loppuun. Tässä 23 sivua

%\thispagestyle{empty}  % nimiölehdellä ei ole sivunumerointia; tyylin mukaan ei tehdäkään?!

\maketitle             % tehdään nimiölehti

% -------------- Tiivistelmä / abstract -------------------------------
% Lisää abstrakti kandikielellä (ja halutessasi lisäksi englanniksi).

% Edelleen sivunumerointiin. Eräs ohje käskee aloittaa sivunumeroiden
% laskemisen nimiösivulta kuitenkin niin, että sille ei numeroa merkitä
% (Kauranen, luku 5.2.2). Näin ollen ensimmäisen tiivistelmän sivunumero
% on 2. \maketitle komento jotenkin kadottaa sivunumeronsa.
\setcounter{page}{2}    % sivunumeroksi tulee 2

% \input{luku_abstraktit}
\newpage                       % pakota sivunvaihto

% -------------- Sisällysluettelo / TOC -------------------------------

\tableofcontents

\label{pages:prelude}
\clearpage                     % kappale loppuu, loput kelluvat tänne, sivunv.
%\newpage

% -------------- Symboli- ja lyhenneluettelo -------------------------
% Lyhenteet, termit ja symbolit.
% Suositus: Käytä vasta kun paljon symboleja tai lyhenteitä.
%
% \input{luku_lyhenteet} 
%\clearpage                     % luku loppuu, loput kelluvat tänne
\newpage

% -------------- Kuvat ja taulukot ------------------------------------
% Kirjoissa (väitöskirja) on usein tässä kuvien ja taulukoiden listaus.
% Suositus: Ei kandityöhön.

% -------------- Alkusanat --------------------------------------------
% Suositus: ÄLÄ käytä kandidaatintyössä. Jos käytät, niin omalle 
% sivulleen käyttäen tarvittaessa \newpage
%
% \section{Introduction}
\subsection{Background}

Robotic motion planning is a field of research in theoretical computer science which has been deeply studied. It focuses on designing and studying algorithms for making robots get from one point to another safely (without colliding) and efficiently (minimizing some properly defined cost function) \cite{mit/chosetPrinciplesRobotMotion2005}.

Robots are modeled in a vast number of ways, which differ in terms of shapes, sizes and kinematics: some warehouse robots can be modeled to move in two dimensions only, while flying drones can be seen as moving in three dimensions.

Robotic arms that move an end-effector are often modeled in six dimensions: three spatial ones and three for orientation. These robotic arms often have so-called kinematic redundancy, meaning they have more degrees of freedom than strictly necessary \cite{robo/ChiaveriniOM16}. Kinematic redundancy increases the fault-tolerance and dexterity of the robot, and makes it possible to re-adjust while hold the same end-effector pose, but it also increases the complexity in motion planning as poses can have different alternative joint configurations \cite{robo/ChiaveriniOM16}. Kinematic redundancy enables the arm to re-adjust while holding the same end-effector pose, which helps dexterity and fault-tolerance, but also increases complexity in planning, as poses can have different alternative joint configurations \cite{robo/ChiaveriniOM16}.

There are many different environments in which robots exist. Assembly line robots have a fairly static environment, well defined start and end positions, and a clear and safe path between them. Robotic vacuum cleaners on the other hand need to map their environment and make decisions dynamically, as furniture, people and pets can change places from time to time. Some robots need to work together; multi-robot systems can be found in warehouses \cite{robo/ParkerRS16}, where parallel motion is highly desirable, but past research has focused mostly on algorithms moving robots one-by-one \cite{siamcomp/DemaineFKMS19}.

\subsection{Parallel robots}

The field of coordinated motion planning for parallel robotics has seen a lot of research in recent years, with algorithms tackling different problems in the field. 

The many real world scenarios to be studied led to the formulation of different problems. The goal is to move the robots from their starting positions to their target positions without colliding with other robots. Without accounting for collisions, finding a solution would of course be trivial, as robots could move straight to their targets. There are two different general types of motion problems: finding a collision-free schedule transforming a workspace from a starting configuration to a target configuration is called a \emph{motion planning problem}, while determining if such a schedule exists is called the \emph{mover's problem} \cite{siamcomp/HopcroftW86}. 

After executing a valid schedule, all robots will have moved from their starting positions to their target positions, and all target positions are occupied by a robot.

Some variations to this motion problem exist: in its \emph{labeled} formulation, robots are unique, and are assigned specific target locations to move to. This problem formulation is the most studied \cite{fun/BrockenHKLS21}. In its \emph{unlabeled} formulation, robots are instead indistinguishable from each other, and targets can be occupied by any robot. In 2014, \cite{ijrr/SoloveyH14} introduced a generalization of labeled and unlabeled problems: in a \emph{(k-)colored} problem, the robots are partitioned into $k$ colors or groups, and each have to move to a target location with the same coloring. \emph{Labeled} and \emph{unlabeled} are then extremes of the \emph{colored} problem formulation: $k=1$ gives an unlabeled problem, as all robots are of the same color, while $k=n$ gives all robots their own unique color. 

% The problem has multiple variations: \refactor{we}{something else}
% in the unlabeled formulation of the problem, we do not care which robot gets to which target position, only that all target positions get occupied. 
% \TODO in the labeled formulation, robots and target positions have labels...
% In case we want to move specific robots to specific targets, we have a labeled problem, where robots and target positions each have labels, such that target positions contain a robot with the same label at the end. 
% Note that such labels can be unique or not, and some works distinguish between these cases: for example, \cite{demaineCoordinatedMotionPlanning2019} considers \emph{unlabeled, colored} and \emph{labeled} as variations on the problem, where \emph{labeled} means unique labels, and \emph{colored} is a generalization of the unlabeled and labeled variants: multiple robots (and targets) can have the same color, and the targets are only to be matched up with robots of the same color. \TODO explicit extreme
% Thus both the unlabeled and labeled formulations can be modeled as specific cases of the colored one.

Real robots are physically three-dimensional objects, but they are mostly restricted to two dimensions in multi-robot applications. There are some works exploring higher dimensions: \cite{arobots/TurpinMMK14}, but current works are quite focused on the two-dimensional setting. Reasons for this might be ease of visualization and thus understanding, simpler algorithms, real world applicability or something else.

The problem can also be modeled both in the continuous or discrete space. In a discrete model, the workspace is modeled as a grid, which can then be handled using graph theory, with grid cells being nodes that the robots occupy and move through.

In a continuous model, the robots can in theory be modeled as close to reality as wanted, but the increased complexity seems to not be worth it. Most current research in motion planning consider only simple shapes, and real warehouse robots also seem to reflect this in their shapes: many are shaped close to circles or squares. Unit radius disks (see \cite{siamcomp/DemaineFKMS19}, \cite{compgeom/BanyassadyBBBFH22}) or unit squares (see \cite{jea/YangV22}) make collision checking a lot simpler: a single distance metric on two robots' centers determines if they are in collision or not. Most continuous space research also study uniform sizes and shapes of robots, but for example \cite{fun/BrockenHKLS21} investigates motion planning of non-uniformly-sized discs.

% Finding the minimum makespan solution to a discrete grid case of the coordinated motion planning problem is found to be NP-hard by \cite{siamcomp/DemaineFKMS19}. \cite{siamcomp/DemaineFKMS19} also conjectures that the continuous formulation of the problem would be at least as hard, but leaves it for future work.
% NP-hardness implies approximation algorithms are justified \cite{siamcomp/DemaineFKMS19}, so research is focused on better approximations and faster computation. 

There is also a distinction between centralized and distributed computing in motion planning. Robots are inherently physically distributed, and a lot of research considers multi-robot systems as a problem of distributed computing. Most research in specifically robotic motion planning uses centralized planning though, where a single entity computes all moves for all robots. Distributed computing considers similar but distinct problems, like \emph{rendezvous} and \emph{exploration}, which are covered in \cite{lncs/FlocchiniGN19}.






\subsection{Thesis objective}

The scope is broad and the problems are varied but related. The goal is to understand and explain current research, what has been done and what is yet to be understood. To achieve this, current state-of-the-art algorithms are compared in terms of their performance in computation and execution, strengths, weaknesses and scalability. How many robots can we support in a real system before motion planning becomes too slow? 

\clearpage                     % luku loppuu, loput kelluvat tänne
\newpage                       % pakota sivunvaihto
%
%SH: Alkusanoissa voi kiittää tahoja, jotka ovat merkittävästi edistäneet
% työn valmistumista. Tällaisia voivat olla esimerkiksi yritys, jonka
% tietokantoja, kontakteja tai välineistöä olet saanut käyttöösi,
% haastatellut henkilöt, ohjaajasi tai muut opettajat ja myös
% henkilökohtaiset kontaktisi, joiden tuki on ollut korvaamatonta työn
% kirjoitusvaiheessa. Alkusanat jätetään tyypillisesti pois
% kandidaatintyöstä, joka on laajuudeltaan vielä niin suppea, ettei
% kiiteltäviä tahoja luontevasti ole.

% ---------------------------------------------------------------------
% -------------- Tekstiosa alkaa --------------------------------------
% ---------------------------------------------------------------------

% Muutetaan tarvittaessa ala- ja ylätunnisteet
%\pagestyle{headings}          % headeriin lisätietoja
%\pagestyle{fancyheadings}     % headeriin lisätietoja
%\pagestyle{plain}             % ei header, footer: sivunumero

% Sivunumerointi, jos käytetty 'roman' aiemmin
% \pagenumbering{arabic}        % 1,2,3, samalla alustaa laskurin ykköseksi
% \thispagestyle{empty}         % pyydetty ensimmäinen tekstisivu tyhjäksi

% input-komento upottaa tiedoston 
% \input{luku_sisalto}
\section{Introduction}
\subsection{Background}

Robotic motion planning is a field of research in theoretical computer science which has been deeply studied. It focuses on designing and studying algorithms for making robots get from one point to another safely (without colliding) and efficiently (minimizing some properly defined cost function) \cite{mit/chosetPrinciplesRobotMotion2005}.

Robots are modeled in a vast number of ways, which differ in terms of shapes, sizes and kinematics: some warehouse robots can be modeled to move in two dimensions only, while flying drones can be seen as moving in three dimensions.

Robotic arms that move an end-effector are often modeled in six dimensions: three spatial ones and three for orientation. These robotic arms often have so-called kinematic redundancy, meaning they have more degrees of freedom than strictly necessary \cite{robo/ChiaveriniOM16}. Kinematic redundancy increases the fault-tolerance and dexterity of the robot, and makes it possible to re-adjust while hold the same end-effector pose, but it also increases the complexity in motion planning as poses can have different alternative joint configurations \cite{robo/ChiaveriniOM16}. Kinematic redundancy enables the arm to re-adjust while holding the same end-effector pose, which helps dexterity and fault-tolerance, but also increases complexity in planning, as poses can have different alternative joint configurations \cite{robo/ChiaveriniOM16}.

There are many different environments in which robots exist. Assembly line robots have a fairly static environment, well defined start and end positions, and a clear and safe path between them. Robotic vacuum cleaners on the other hand need to map their environment and make decisions dynamically, as furniture, people and pets can change places from time to time. Some robots need to work together; multi-robot systems can be found in warehouses \cite{robo/ParkerRS16}, where parallel motion is highly desirable, but past research has focused mostly on algorithms moving robots one-by-one \cite{siamcomp/DemaineFKMS19}.

\subsection{Parallel robots}

The field of coordinated motion planning for parallel robotics has seen a lot of research in recent years, with algorithms tackling different problems in the field. 

The many real world scenarios to be studied led to the formulation of different problems. The goal is to move the robots from their starting positions to their target positions without colliding with other robots. Without accounting for collisions, finding a solution would of course be trivial, as robots could move straight to their targets. There are two different general types of motion problems: finding a collision-free schedule transforming a workspace from a starting configuration to a target configuration is called a \emph{motion planning problem}, while determining if such a schedule exists is called the \emph{mover's problem} \cite{siamcomp/HopcroftW86}. 

After executing a valid schedule, all robots will have moved from their starting positions to their target positions, and all target positions are occupied by a robot.

Some variations to this motion problem exist: in its \emph{labeled} formulation, robots are unique, and are assigned specific target locations to move to. This problem formulation is the most studied \cite{fun/BrockenHKLS21}. In its \emph{unlabeled} formulation, robots are instead indistinguishable from each other, and targets can be occupied by any robot. In 2014, \cite{ijrr/SoloveyH14} introduced a generalization of labeled and unlabeled problems: in a \emph{(k-)colored} problem, the robots are partitioned into $k$ colors or groups, and each have to move to a target location with the same coloring. \emph{Labeled} and \emph{unlabeled} are then extremes of the \emph{colored} problem formulation: $k=1$ gives an unlabeled problem, as all robots are of the same color, while $k=n$ gives all robots their own unique color. 

% The problem has multiple variations: \refactor{we}{something else}
% in the unlabeled formulation of the problem, we do not care which robot gets to which target position, only that all target positions get occupied. 
% \TODO in the labeled formulation, robots and target positions have labels...
% In case we want to move specific robots to specific targets, we have a labeled problem, where robots and target positions each have labels, such that target positions contain a robot with the same label at the end. 
% Note that such labels can be unique or not, and some works distinguish between these cases: for example, \cite{demaineCoordinatedMotionPlanning2019} considers \emph{unlabeled, colored} and \emph{labeled} as variations on the problem, where \emph{labeled} means unique labels, and \emph{colored} is a generalization of the unlabeled and labeled variants: multiple robots (and targets) can have the same color, and the targets are only to be matched up with robots of the same color. \TODO explicit extreme
% Thus both the unlabeled and labeled formulations can be modeled as specific cases of the colored one.

Real robots are physically three-dimensional objects, but they are mostly restricted to two dimensions in multi-robot applications. There are some works exploring higher dimensions: \cite{arobots/TurpinMMK14}, but current works are quite focused on the two-dimensional setting. Reasons for this might be ease of visualization and thus understanding, simpler algorithms, real world applicability or something else.

The problem can also be modeled both in the continuous or discrete space. In a discrete model, the workspace is modeled as a grid, which can then be handled using graph theory, with grid cells being nodes that the robots occupy and move through.

In a continuous model, the robots can in theory be modeled as close to reality as wanted, but the increased complexity seems to not be worth it. Most current research in motion planning consider only simple shapes, and real warehouse robots also seem to reflect this in their shapes: many are shaped close to circles or squares. Unit radius disks (see \cite{siamcomp/DemaineFKMS19}, \cite{compgeom/BanyassadyBBBFH22}) or unit squares (see \cite{jea/YangV22}) make collision checking a lot simpler: a single distance metric on two robots' centers determines if they are in collision or not. Most continuous space research also study uniform sizes and shapes of robots, but for example \cite{fun/BrockenHKLS21} investigates motion planning of non-uniformly-sized discs.

% Finding the minimum makespan solution to a discrete grid case of the coordinated motion planning problem is found to be NP-hard by \cite{siamcomp/DemaineFKMS19}. \cite{siamcomp/DemaineFKMS19} also conjectures that the continuous formulation of the problem would be at least as hard, but leaves it for future work.
% NP-hardness implies approximation algorithms are justified \cite{siamcomp/DemaineFKMS19}, so research is focused on better approximations and faster computation. 

There is also a distinction between centralized and distributed computing in motion planning. Robots are inherently physically distributed, and a lot of research considers multi-robot systems as a problem of distributed computing. Most research in specifically robotic motion planning uses centralized planning though, where a single entity computes all moves for all robots. Distributed computing considers similar but distinct problems, like \emph{rendezvous} and \emph{exploration}, which are covered in \cite{lncs/FlocchiniGN19}.






\subsection{Thesis objective}

The scope is broad and the problems are varied but related. The goal is to understand and explain current research, what has been done and what is yet to be understood. To achieve this, current state-of-the-art algorithms are compared in terms of their performance in computation and execution, strengths, weaknesses and scalability. How many robots can we support in a real system before motion planning becomes too slow? 

\clearpage                     % luku loppuu, loput kelluvat tänne, sivunv.

%\input{luku2}                  % tässä tyylissä ei sivunvaihtoja lukujen
%\input{luku3}                  %   välillä. Toiset ohjaajat haluavat 
%\input{luku4}                  %   sivunvaihdot.

\label{pages:text}
\clearpage                     % luku loppuu, loput kelluvat tänne, sivunvaihto
%\newpage                       % ellei ylempi tehoa, pakota lähdeluettelo 
                               % alkamaan uudelta sivulta

% -------------- Lähdeluettelo / reference list -----------------------
%
% Lähdeluettelo alkaa aina omalta sivultaan; pakota lähteet alkamaan
% joko \clearpage tai \newpage
%
%
% Muista, että saat kirjallisuusluettelon vasta
%  kun olet kääntänyt ja kaulinnut "latex, bibtex, latex, latex"
%  (ellet käytä Makefilea ja "make")

% Viitetyylitiedosto aaltosci_t.bst; muokattu HY:n tktl-tyylistä.
\bibliographystyle{aaltosci_t}
% Katso myös tämän tiedoston yläosan "preamble" ja siellä \bibpunct.

% Muutetaan otsikko "Kirjallisuutta" -> "Lähteet"
\renewcommand{\refname}{\REFERENCES}  % article-tyyppisen
%\renewcommand{\bibname}{Lähteet}  % jos olisi book, report-tyyppinen

% Lisätään sisällysluetteloon
\addcontentsline{toc}{section}{\refname}  % article
%\addcontentsline{toc}{chapter}{\bibname}  % book, report

% Määritä kaikki bib-tiedostot
\bibliography{zotero}
%\bibliography{thesis_sources,ietf_sources}

\label{pages:refs}
\clearpage         % erotetaan mahd. liitteet alkamaan uudelta sivulta

% -------------- Liitteet / Appendices --------------------------------
%
% Liitteitä ei yleensä tarvita. Kommentoi tällöin seuraavat
% rivit.

% Tiivistelmässä joskus matemaattisen kaavan tarkempi johtaminen, 
% haastattelurunko, kyselypohja, ylimääräisiä kuvia, lyhyitä 
% ohjelmakoodeja tai datatiedostoja.

\appendix
% \input{luku_liitteet}

\label{pages:appendices}

% ---------------------------------------------------------------------

\end{document}
