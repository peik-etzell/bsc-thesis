\section{Introduction}
\subsection{Background}

Robotic motion planning is a field of research in theoretical computer science which has been extensively studied.
It focuses on designing and studying algorithms that get robots from one point to another (without colliding) and efficiently (minimizing some properly defined cost function) \cite{mit/chosetPrinciplesRobotMotion2005}.

Robots are modeled in a vast number of ways, which differ in terms of shapes, sizes and kinematics: some warehouse robots can be modeled to move in two dimensions only, while flying drones can be seen as moving in three dimensions.

Robotic arms that move an end-effector are often modeled in six dimensions: three spatial ones and three for orientation.
These robotic arms often have so-called kinematic redundancy, meaning they have more degrees of freedom than strictly necessary \cite{robo/ChiaveriniOM16}.
Kinematic redundancy increases the fault-tolerance and dexterity of the robot, and makes it possible to re-adjust while hold the same end-effector pose, but it also increases the complexity in motion planning as poses can have different alternative joint configurations \cite{robo/ChiaveriniOM16}.
Kinematic redundancy enables the arm to re-adjust while holding the same end-effector pose, which helps dexterity and fault-tolerance, but also increases complexity in planning, as poses can have different alternative joint configurations \cite{robo/ChiaveriniOM16}.

There are many different environments in which robots exist.
For instance, assembly line robots have a fairly static environment, well defined start and end positions, and a clear and safe path between them.
Robotic vacuum cleaners need to map their environment and make decisions dynamically, as furniture, people and pets can change places from time to time.
Some robots need to work together; multi-robot systems can be found in warehouses \cite{robo/ParkerRS16}, where parallel motion is highly desirable, but past research has focused mostly on algorithms moving robots one-by-one \cite{siamcomp/DemaineFKMS19}.

\subsection{Parallel robots}

The field of coordinated motion planning for parallel robotics has seen a lot of research in recent years, with algorithms tackling different problems in the field.


The many real world scenarios to be studied led to the formulation of different problems.
The goal is to move the robots from their starting positions to their target positions without colliding with other robots.
Without accounting for collisions, finding a solution would of course be trivial, as robots could move straight to their targets.
There are two different general types of motion problems: finding a collision-free schedule transforming a workspace from a starting configuration to a target configuration is called a \emph{motion planning problem}, while determining if such a schedule exists is called the \emph{mover's problem} \cite{siamcomp/HopcroftW86}.

Some variations to this motion problem exist: in its \emph{labeled} formulation, robots are unique, and are assigned specific target locations to move to.
This problem formulation is the most studied \cite{fun/BrockenHKLS21}.
In its \emph{unlabeled} formulation, robots are instead indistinguishable from each other, and targets can be occupied by any robot.
In 2014, \cite{ijrr/SoloveyH14} introduced a generalization of labeled and unlabeled problems: in a \emph{(k-)colored} problem, the robots are partitioned into $k$ colors or groups, and each have to move to a target location with the same coloring.
\emph{Labeled} and \emph{unlabeled} are then extremes of the \emph{colored} problem formulation: $k=1$ gives an unlabeled problem, as all robots are of the same color, while $k=n$ gives all robots their own unique color.

After executing a valid schedule, all robots will have moved from their starting positions to their target positions, and all target positions are occupied by a robot.
There is naturally a desire to execute the schedule \emph{efficiently} by some metric.
We will mainly focus on the \emph{makespan} in this thesis: the total time taken for all robots to get to their target positions.
Other common metrics are \emph{total arrival time}, \emph{total distance} and \emph{maximum distance}; more on these in \cref{chapter:simultaneous_optimization}. 

Real robots are physically three-dimensional objects, but they are mostly restricted to two dimensions in multi-robot applications.
There are some works exploring higher dimensions: \cite{arobots/TurpinMMK14}, but current works are quite focused on the two-dimensional setting.
Reasons for this might be ease of visualization and thus understanding, simpler algorithms, real world applicability or something else.

The problem can also be modeled both in the continuous or discrete space.
In a discrete model, the workspace is modeled as a grid, which can then be handled using graph theory, with grid cells being nodes that the robots occupy and move through.

In a continuous model, the robots can in theory be modeled as close to reality as wanted, but the increased complexity seems to not be worth it.
Most current research in motion planning consider only simple shapes, and real warehouse robots also seem to reflect this in their shapes: many are shaped close to circles or squares.
Unit radius disks (see \cite{siamcomp/DemaineFKMS19}, \cite{compgeom/BanyassadyBBBFH22}) or unit squares (see \cite{jea/YangV22}) make collision checking a lot simpler: a single distance metric on two robots' centers determines if they are in collision or not.
Most continuous space research also study uniform sizes and shapes of robots, but for example \cite{fun/BrockenHKLS21} investigates motion planning of non-uniformly-sized discs.

There is also a distinction between centralized and distributed computing in motion planning.
Robots are inherently physically distributed, and a lot of research considers multi-robot systems as a problem of distributed computing.
Most research in specifically robotic motion planning uses centralized planning though, where a single entity computes all moves for all robots.
Distributed computing considers similar but distinct problems, like \emph{rendezvous} and \emph{exploration}, which are covered in \cite{lncs/FlocchiniGN19}.




\subsection{Thesis contributions}

This thesis focuses specifically on centralized motion planning for parallel robotics on the planar grid.
The goal of this thesis is to understand and explain the current research in the field, what has been done and what is yet to be understood.

In \cref{chapter:main_proof} we prove the computational complexity of low-makespan parallel motion planning by a simplified reduction from one originally constructed in \cite{siamcomp/DemaineFKMS19}.
This proof shows that the study of approximation algorithms is justified in the field.
In \cref{chapter:constant_stretch} we look at two recent approximation algorithms from \cite{siamcomp/DemaineFKMS19}, which can efficiently compute low-makespan schedules for any problem on a grid.
Lastly, in \cref{chapter:simultaneous_optimization} we show that one cannot always have a schedule optimizing any pair of the four different cost functions, including the makespan, that we define.
This was previously proven to be true using non-grid graphs by \cite{corr/YuL15c}, we extend this proof to hold specifically on the grid.
