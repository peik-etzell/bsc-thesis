\section{Conclusions}
\fda{develop section as discussed in the meeting}


We have looked at several important aspects of coordinated multi-robot motion planning on the grid:

In \cref{chapter:main_proof} we proved that the decision problem of minimum makespan motion planning is NP-complete using a simplified version of a reduction from \cite{siamcomp/DemaineFKMS19}. 
The proof uses substantially fewer robots than the original, which should make it easier to grasp. 
The proof is important in that it shows the value of efficient approximation algorithms in the field of parallel robotics. 

In \cref{chapter:constant_stretch} we took a high-level look at some of the approximation algorithms introduced by \cite{siamcomp/DemaineFKMS19}, which can efficiently compute schedules with constant stretch, implying a guaranteed level of robot performance even in large problem instances.

Lastly, in \cref{chapter:alternative_metrics} we proved that a theorem from \cite{corr/YuL15c} stated for general graphs also holds specifically on the grid. 
The theorem shows the importance of choosing the right cost function when optimizing solutions to motion planning problems. 

Only a small part of the field of coordinated motion planning was covered though. 
Still left untouched are various different problem formulations, especially motion planning modeled in the continuous space seems central for future research.
