\section{The general minimum makespan parallel motion planning problem on a grid is NP-hard}

\cite{siamcomp/DemaineFKMS19} \cite{corr/YuL15c}

\subsection{Preliminaries}

A \emph{Turing Machine} (TM) is a mathematically modeled machine capable of general purpose computations operating on a tape of symbols. It is generally considered to be equivalent in capabilities to most mathematical definitions of computation \cite{aw/HopcroftU79}.

There are different \emph{classes} of computational problems in terms of computational complexity. Any problem belonging to the \emph{P}-class of problems is solvable by a TM in \emph{polynomial time}, i.e. in \ilmath{O(n^k)} time on the size of the input \emph{n} and some constant \emph{k}.

A problem in the \emph{NP}-class of problems is solvable in \emph{nondeterministic polynomial time}. It means a correct solution is \emph{verifiable} in polynomial time, but finding a solution might take considerably longer. It is an important open question whether \ilmath{P = NP}, but most mathematicians believe P is a proper subset of NP.

\note{NP-complete}

\begin{definition}
	A problem is classified as \emph{NP-hard} if it is \emph{at least as hard} as the hardest problems in the \emph{NP}-class. 
\end{definition}

A computational problem that asks a yes/no question based on some input is called a \emph{decision problem}.

\begin{definition}
	3SAT
\end{definition}

\subsection{The theorem}

\begin{theorem}
	The problem of computing a schedule with minimum makespan to a colored motion planning problem on a grid is NP-hard.
\end{theorem}

\begin{proof}
	\dots
\end{proof}
