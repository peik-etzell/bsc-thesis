
\section{Conclusions}

We have looked at several important aspects of coordinated multi-robot motion planning on the grid:

In \cref{chapter:main_proof} we proved that the decision problem of minimum makespan motion planning is NP-complete using an easier version of a reduction from \cite{siamcomp/DemaineFKMS19}.
The proof uses substantially fewer robots than the original, which makes it easier to grasp. 
% The proof is important in that it shows the value of efficient approximation algorithms in the field of parallel robotics. \note{rephrase}

In \cref{chapter:constant_stretch} we took a high-level look at some of the approximation algorithms introduced by \cite{siamcomp/DemaineFKMS19}, which can efficiently compute schedules with constant approximation factor, implying a guaranteed level of robot performance even in large problem instances.

Lastly, in \cref{chapter:simultaneous_optimization} we extended a theorem from \cite{corr/YuL15c} to hold specifically on the grid: considering four distinct cost functions, any pair cannot always be simultaneously optimized.
This shows that choosing different cost functions to optimize yield different outputs.

% Only a small part of the field of coordinated motion planning was covered though.
% There are various different problem formulations that follow the real world closer.

Although the scope of this thesis is limited, it has given insight into the field of coordinated motion planning on a discrete grid.
More efficient schedules are possible in the continuous case for example, but this domain is harder and less well-understood than the discrete case.
It is conjectured by \cite{siamcomp/DemaineFKMS19} that the optimal makespan motion planning problem in a continuous model would be harder than the discretely modeled problem. 

Schedules for the discrete space can of course be executed with real robots, assuming that they are sufficiently sparsely positioned to enable movements in discrete steps. 
Thus theoretical work in the discrete space can in many cases be directly transferred to real world implementations.

