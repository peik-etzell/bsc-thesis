\documentclass[12pt,a4paper,english,oneside]{article}

% Valitse 'input encoding':
%\usepackage[latin1]{inputenc} % merkistökoodaus, jos ISO-LATIN-1:tä.
\usepackage[utf8]{inputenc}   % merkistökoodaus, jos käytetään UTF8:a
% Valitse 'output/font encoding':
%\usepackage[T1]{fontenc}      % korjaa ääkkösten tavutusta, bittikarttana
\usepackage{ae,aecompl}       % ed. lis. vektorigrafiikkana bittikartan sijasta
% Kieli- ja tavutuspaketit:
\usepackage[english]{babel}
% Muita paketteja:
% \usepackage{amsmath}   % matematiikkaa
\usepackage{url}       % \url{...}

% Kappaleiden erottaminen ja sisennys
\parskip 1ex
\parindent 0pt
\evensidemargin 0mm
\oddsidemargin 0mm
\textwidth 159.2mm
\topmargin 0mm
\headheight 0mm
\headsep 0mm
\textheight 246.2mm

\pagestyle{plain}

% ---------------------------------------------------------------------

\begin{document}

% Otsikkotiedot: muokkaa tähän omat tietosi

\title{CS BSc thesis research plan:\\[5mm]
Robotic Motion Planning}

\author{Peik Etzell\\
Aalto-university\\
\url{peik.etzell@aalto.fi}}

\date{\today}

\maketitle

% ---------------------------------------------------------------------

\vspace{10mm}

\textbf{Thesis title:} Robotic motion planning: An overview % TODO

\textbf{Author:} Peik Etzell

\textbf{Supervisor:} Francesco d'Amore


\section{Introduction}

Robotic motion planning is a very broad topic, generally centered around planning sequences of moves for making robots move from one point to another, safely and efficiently. 

This bachelors thesis will focus on existing algorithmic research in coordinated multi-robot motion planning.
Finding an optimal solution for these problems is found to be NP-hard \cite{demaine_coordinated_2019}. NP-hardness implies approximation algorithms are justified \cite{demaine_coordinated_2019}, so research is focused on better approximations and faster computation. 

Different problem statements exist for multi-robot motion planning. Robots, or agents, either move in a grid, or they can move continuously in two-dimensional space. 
Grids can be modelled with graph theory, such that a node can only be occupied by a single agent at any time. 
Continuous spaces need to use some kind of geometry to avoid collisions, and thus the agents are often modeled as unit discs in the plane. 

There also exist different variations on the problem in the way that the agents can map onto the targets. In an unlabeled case, the agents are indistinguishable, and all target positions can be occupied by any agent. In a case with colored agents, the target positions each want a specific color of agent. Lastly, in a labeled case, all agents have a unique label and target position, equivalent to each having their own color in a colored case.

The subject is only a small part of the vast field of robotics, with problems touching on many fields of engineering. 
With the subject being so closely tied to the real world with physical robots, motion planning itself is not enough to make a robot move effectively. 
A closely related topic is \emph{inverse and forwards kinematics}, which translate between so-called \emph{joint-space} and the real world, and is essential if the robot motions are more complex than simple linear movements. 

\section{Research objectives}

The objective with this bachelors thesis is to dig into current research in coordinated motion planning, and give an overview of the field. Here are some of the central questions that I hope to answer: 

\begin{itemize}
	\item What are the different problems in coordinated multi-robot motion planning?
	\item How does the best current algorithms work?
	\item Are they applicable to the real world?
	\item What are the limitations, how many robots can feasibly be planned for?
	\item Real time or batch processing? 
\end{itemize}


\section{Material and methods}

This is a theoretical review of existing research, and will not use any physical materials. 

Research papers, textbooks etc. from reputable sources will be used to write this bachelors thesis.

Three journal articles were provided in the topic listing:

\begin{itemize}
	\item Unlabeled Multi-Robot Motion Planning with Tighter Separation Bounds \cite{banyassady_unlabeled_2022}
	\item Coordinated Motion Planning: Reconfiguring a Swarm of Labeled Robots with Bounded Stretch \cite{demaine_coordinated_2019}.
	\item Space-Aware Reconfiguration \cite{halperin_space-aware_2022}
\end{itemize}

Some additional sources that have been identified at this point:

\begin{itemize}
	\item Reducing Multiple Object Motion Planning to Graph Searching \cite{hopcroft_reducing_1986} 
	\item On the complexity of motion planning for multiple independent objects; PSPACE-hardness of the "warehouseman's problem" (no pdf found at the moment, contact library)
	\item Principles of Robot Motion: Theory, Algorithms and Implementation \cite{choset_principles_2005}
	\item Distributed Computing by Mobile Entities: Current Research in Moving and Computing \cite{flocchini_distributed_2019}
	\item Motion Planning and Reconfiguration for Systems of Multiple Objects \cite{kolski_motion_2007}
\end{itemize}

More sources will be found mostly by examining the references and citations of found resources. 

\section{Schedule}

Here are the major deadlines of the course and some notes:

\begin{tabular}{|p{50mm}|p{30mm}|p{65mm}|}
	\hline
	Research plan (this)   & 30.1 at 12.00 & Usable introduction
	\\ \hline
	Version 1   & 13.2 at 12.00 & Mostly reading, some text
	\\ \hline
	Peer feedback on V1 & 17.2 at 12.00 & 
	\\ \hline
	Version 2   & 6.3 at 12.00 & Writing well under way
	\\ \hline
	Version 3   & 27.3 at 12.00 & All content written
	\\ \hline
	Written opposition of V3 & 31.3 at 12.00 & 
	\\ \hline
	Taking a break   & ca. 5.4 --- 12.4 & Hopefully finish V4 before this
	\\ \hline
	Version 4   & 17.4 at 12.00 & Final adjustments
	\\ \hline
	Presentations  	& 2.5 --- 5.5 & Presentation specifics
	\\ \hline
\end{tabular}


% ---------------------------------------------------------------------
%
% ÄLÄ MUUTA MITÄÄN TÄÄLTÄ LOPUSTA

% Tässä on käytetty siis numeroviittausjärjestelmää. 
% Toinen hyvin yleinen malli on nimi-vuosi-viittaus.

% \bibliographystyle{plainnat}
% \bibliographystyle{finplain}  % suomi
\bibliographystyle{plain}    % englanti
% Lisää mm. http://amath.colorado.edu/documentation/LaTeX/reference/faq/bibstyles.pdf

% Muutetaan otsikko "Kirjallisuutta" -> "Lähteet"
\renewcommand{\refname}{References}  % article-tyyppisen

% Määritä bib-tiedoston nimi tähän (eli lahteet.bib ilman bib)
% \bibliography{lahteet}
\bibliography{zotero}

% ---------------------------------------------------------------------

\end{document}
